\input{preamble/preamble.tex}
\input{preamble/definitions.tex}

\everymath{\displaystyle}
\setcounter{chapter}{1}

\begin{document}


\begin{center}
    \minibox[frame,c,pad=5pt]{\large \bfseries  Σειρές \\ \large Ασκήσεις}
\end{center}

\vspace{\baselineskip}

\begin{enumerate}
    \item 
    \begin{enumerate}[i)]
        \item Έχουμε $ a_{n}= \frac{n^{2}+1}{3n^{2}-1} > 0, \; \forall n \in 
            \mathbb{N} $. Επομένως η σειρά συγκλίνει ή απειρίζεται θετικά.

            Όμως \[ \lim_{n \to \infty} a_{n} = \lim_{n \to \infty} 
                \frac{n^{2}+1}{3n^{2}-1} = \lim_{n \to \infty} 
                \frac{n^{2}(1+ \frac{1}{n^{{2}}})}{n^{2}(3- \frac{1}{n^{2}})} = 
                \lim_{n \to \infty} \frac{1 + \frac{1}{n^{2}}}{3- \frac{1}{n^{2}}} = 
                \frac{1+0}{3-0} = \frac{1}{3} \neq 0
            \] επομένως από γνωστή πρόταση η σειρά αποκλίνει και άρα απειρίζεται 
            θετικά.

        \item Έχουμε $ a_{n}= \frac{2n^{3}+n-1}{n^{3}+4} > 0, \; \forall n \in 
            \mathbb{N} $. Επομένως η σειρά συγκλίνει ή απειρίζεται θετικά.
            Όμως
            \[
                \lim_{n \to \infty} \frac{2n^{3}+n-1}{n^{3}+4} = \lim_{n \to \infty}
                \frac{n^{3}(2+ \frac{1}{n^{2}}- \frac{1}{n^{3}})}{n^{3}
                (1+ \frac{4}{n^{3}})} = \lim_{n \to \infty} = 
                \lim_{n \to \infty} \frac{2+ \frac{1}{n^{2}}- 
                \frac{1}{n^{3}}}{1 + \frac{4}{n^{3}}} =  
                \frac{2+0-0}{1+0} = 2 \neq 0
            \]
            επομένως από γνωστή πρόταση η σειρά αποκλίνει και 
            άρα απειρίζεται θετικά. 
    \end{enumerate}

\item 
    \begin{enumerate}[i)]
        \item Παρατηρούμε ότι $ a_{n} = \frac{n^{2}+1}{n} \geq 0, \; \forall n 
            \in \mathbb{N} $. Έχουμε 
            \[
                a_{n} = \frac{n^{2}+1}{n} = n + \frac{1}{n} \geq 
                \frac{1}{n}, \; \forall n \in \mathbb{N}  
            \]
            Επομένως $ 0 \leq \frac{1}{n} \leq \frac{n^{2}+1}{n}, \; \forall n \in
            \mathbb{N}  $ και $ \sum_{n=1}^{\infty} \frac{1}{n} $ αποκλίνει, επομένως 
            από κριτήριο σύγκρισης και $ \sum_{n=1}^{\infty} \frac{n^{2}+1}{n} $ 
            αποκλίνει.

        \item Παρατηρούμε ότι $ a_{n} = \frac{3n^{3}+1}{4n^{4}-1} \geq 0, \; \forall n 
            \in \mathbb{N} $. Έχουμε
            \[
                a_{n} = \frac{3n^{3}+1}{4n^{4}-1} \geq \frac{3n^{3}}{4n^{4}} = 
                \frac{3}{4} \frac{1}{n}, \; \forall n \in \mathbb{N} 
            \] 
            Επομένως $ 0 \leq \frac{3}{4} \frac{1}{n} \leq \frac{3n^{3}+1}{4n^{3}-1}, 
            \; \forall n \in \mathbb{N} $ και $ \sum_{n=1}^{\infty} \frac{3}{4} 
            \frac{1}{n} $ αποκλίνει γιατί $ \sum_{n=1}^{\infty} \frac{1}{n} $ 
            αποκλίνει, επομένως από κριτήριο σύγκρισης και $ \sum_{n=1}^{\infty} 
            \frac{3n^{3}+1}{4n^{4}-1} $ αποκλίνει.

        \item Παρατηρούμε ότι $ a_{n} = \frac{10n+2019}{n^{2}+1} \geq 0, \; 
            \forall n \in \mathbb{N} $. Έχουμε
            \[
                a_{n} = \frac{10n+2019}{n^{2}+1} \geq \frac{10n}{n^{2}+1} \geq 
                \frac{10n}{n^{2}+n^{2}} = \frac{10n}{2n^{2}} = 5 \frac{1}{n}, \; 
                \forall n \in \mathbb{N}
            \] 
            Επομένως $ 0 \leq 5 \frac{1}{n} \leq \frac{10n+2019}{n^{2}+1}, \; 
            \forall n \in \mathbb{N} $ και $ \sum_{n=1}^{\infty} 5 \frac{1}{n} $ 
            αποκλίνει, γιατί $ \sum_{n=1}^{\infty} \frac{1}{n} $ αποκλίνει, 
            επομένως από κριτήριο Σύγκρισης και $ \sum_{n=1}^{\infty} 
            \frac{10n +2019}{n^{2}+1} $ αποκλίνει.

        \item Παρατηρούμε ότι $ a_{n} = \frac{2n}{n^{2}+1} \geq 0, \; \forall n \in 
            \mathbb{N} $.  Έχουμε 
            \[
                a_{n} = \frac{2n}{n^{2}+1} \geq \frac{2n}{n^{2}+n^{2}} = 
                \frac{2n}{2n^{2}} \geq \frac{1}{n}, \; \forall n \in \mathbb{N} 
            \] 
            Επομένως $ 0 \leq \frac{1}{n} \leq \frac{2n}{n^{2}+1}, \; 
            \forall n \in \mathbb{N} $ και $ \sum_{n=1}^{\infty} \frac{1}{n} $ 
            αποκλίνει, επομένως από κριτήριο Σύγκρισης και $ \sum_{n=1}^{\infty} 
            \frac{2n}{n^{2}+1} $ αποκλίνει.

        \item Παρατηρούμε ότι $ a_{n} = \frac{\sqrt[3]{n^{2}+1}}{n+1} \geq 0, \; 
            \forall n \in \mathbb{N} $. Έχουμε
            \[
                a_{n} = \frac{\sqrt[3]{n^{2}+1}}{n+1} \geq \frac{\sqrt[3]{n^{2}}}{n+1} 
                \geq \frac{\sqrt[3]{n^{2}}}{n+n} = \frac{\sqrt[3]{n^{2}}}{2n} =
                \frac{1}{2n^{\frac{1}{3}}}, \; \forall n \in \mathbb{N}
            \] 
            Επομένως $ 0 \leq \frac{1}{2n^{\frac{1}{3}}} \leq 
            \frac{\sqrt[3]{n^{2}+1} }{n+1}, \; \forall n \in \mathbb{N} $ και 
            $ \sum_{n=1}^{\infty} \frac{1}{2} \frac{1}{n^{\frac{1}{3}}} $ 
            αποκλίνει γιατί $ \sum_{n=1}^{\infty} \frac{1}{n^{\frac{1}{3}}} $ 
            αποκλίνει, ως γενικευμένη αρμονική με $ \rho = \frac{1}{3} < 1 $ 
            επομένως από κριτήριο Σύγκρισης και $ \sum_{n=1}^{\infty} 
            \frac{\sqrt[3]{n^{2}+1}}{n+1} $ αποκλίνει.

        \item Παρατηρούμε ότι $ a_{n} = \frac{1}{\sqrt{n(n+1)}} \geq 0, \; \forall n 
            \in \mathbb{N} $. Έχουμε
            \[
                a_{n} = \frac{1}{\sqrt{n(n+1)}} \geq \frac{1}{\sqrt{(n+1)(n+1)}} = 
                \frac{1}{\sqrt{(n+1)^{2}} } = \frac{1}{n+1} \geq \frac{1}{n+n} = 
                \frac{1}{2n}, \; \forall n \in \mathbb{N}   
            \] 
            Επομένως $ 0 \leq \frac{1}{2n} \leq \frac{1}{\sqrt{n(n+1)}}, \; 
            \forall n \in \mathbb{N} $ και $ \sum_{n=1}^{\infty} \frac{1}{2} 
            \frac{1}{n} $ αποκλίνει γιατί $ \sum_{n=1}^{\infty} \frac{1}{n} $ 
            αποκλίνει, επομένως από κριτήριο Σύγκρισης και 
            $ \sum_{n=1}^{\infty} \frac{1}{\sqrt{n(n+1)}} $ αποκλίνει.

            \begin{rem}
                Ένας 2ος τρόπος
                \[
                    a_{n} = \frac{1}{\sqrt{n(n+1)}} = \frac{1}{\sqrt{n^{2}+n}} \geq 
                    \frac{1}{\sqrt{n^{2}+n^{2}}} = \frac{1}{\sqrt{2n^{2}}} = 
                    \frac{1}{\sqrt{2}n}, \; \forall n \in \mathbb{N}   
                \] 
            \end{rem}

        \item Παρατηρούμε ότι $ a_{n} = \frac{\sqrt{n}}{n^{3}+1} \geq 0, \; 
            \forall n \in \mathbb{N} $. Έχουμε
            \[
                a_{n} = \frac{\sqrt{n}}{n^{3}+1} \leq \frac{\sqrt{n}}{n^{3}} =
                \frac{n^{\frac{1}{2}}}{n^{3}} = \frac{1}{n^{\frac{5}{2}}}, \; 
                \forall n \in \mathbb{N}
            \] 
            Επομένως $ 0 \leq \frac{\sqrt{n}}{n^{3}+1} \leq \frac{1}{n^{\frac{5}{2}}}, 
            \; \forall n \in \mathbb{N}$ και $ \sum_{n=1}^{\infty} 
            \frac{1}{n^{\frac{5}{2} }}$ συγκλίνει ως γενικευμένη αρμονική με 
            $ \rho = \frac{5}{2} > 1 $, επομένως από κριτήριο Σύγκρισης η 
            $ \sum_{n=1}^{\infty} \frac{\sqrt{n}}{n^{3}+1} $ συγκλίνει.

        \item Παρατηρούμε ότι $ a_{n} = \frac{\sqrt{n}}{n + 5 \sqrt{n}} \geq 0, \; 
            \forall n \in \mathbb{N}$. Έχουμε
            \[
                a_{n} = \frac{\sqrt{n}}{n+ 5 \sqrt{n}} \geq \frac{\sqrt{n}}{n+ 5n} = 
                \frac{\sqrt{n}}{6n} = \frac{1}{6 \sqrt{n}} \geq \frac{1}{6n}, 
                \; \forall n \in \mathbb{N}
            \] 
            Επομένως $ 0 \leq \frac{1}{6n} \leq \frac{\sqrt{n}}{n+ 5 \sqrt{n}}, \; 
            \forall n \in \mathbb{N} $ και 
            $ \sum_{n=1}^{\infty} \frac{1}{6} \frac{1}{n} $ αποκλίνει γιατί 
            $ \sum_{n=1}^{\infty} \frac{1}{n} $ αποκλίνει, επομένως από κριτήριο 
            Σύγκρισης και $ \sum_{n=1}^{\infty} \frac{\sqrt{n}}{n+ 5 \sqrt{n}} $ 
            αποκλίνει.

        \item Παρατηρούμε ότι $ \sum_{n=1}^{\infty} \frac{3^{n}}{5^{n}+1} \geq 0, 
            \; \forall n \in \mathbb{N}$. Έχουμε
            \[
                a_{n} = \frac{3^{n}}{5^{n}+1} \leq \frac{3^{n}}{5^{n}} = 
                \left(\frac{3}{5}\right)^{n}, \; \forall n \in \mathbb{N} 
            \] 
            Επομένως $ 0 \leq \frac{3^{n}}{5^{n}+1} \leq \left(\frac{3}{5} \right)^{n},
            \; \forall n \in \mathbb{N} $ και $ \sum_{n=1}^{\infty} 
            \left(\frac{3}{5} \right)^{n}$
            συγκλίνει ως γεωμετρική σειρά με $ \abs{\lambda} = \abs{\frac{3}{5}} = 
            \frac{3}{5} < 1 $, επομένως από κριτήριο Σύγκρισης και 
            $ \sum_{n=1}^{\infty} \frac{3^{n}}{5^{n}+1} $ συγκλίνει.

        \item Παρατηρούμε ότι $ \sum_{n=1}^{\infty} \frac{1}{n} 
            \left(\frac{2}{5} \right)^{n} \geq 0, \; \forall n \in \mathbb{N} $. Έχουμε
            \[
                a_{n} = \frac{1}{n} \left(\frac{2}{5} \right)^{n} \leq 1
                \left(\frac{2}{5} \right)^{n}, \; \forall n \in \mathbb{N}
            \] 
            Επομένως $ 0 \leq \frac{1}{n} \left(\frac{2}{5} \right)^{n} \leq 
            \left(\frac{2}{5}\right)^{n}, \; \forall n \in \mathbb{N}$ και 
            $ \sum_{n=1}^{\infty} \left(\frac{2}{5} \right)^{n} $ συγκλίνει 
            ως γεωμετρική με $ \abs{\lambda} = \abs{\frac{2}{5}} = \frac{2}{5} \leq 1$,
            επομένως και $ \sum_{n=1}^{\infty} \frac{1}{n} \left(\frac{2}{5} 
            \right)^{n} $ συγκλίνει.

        \item Παρατηρούμε ότι $ a_{n} = \frac{1}{2^{n}} 
            \abs{\sin{\left(\frac{n^{2}+1}{n+2}\right)}} \geq 0, \; 
            \forall n \in \mathbb{N}$.  Έχουμε
            \[
                a_{n} = \frac{1}{2^{n}} \abs{\sin{\left(\frac{n^{2}+1}{n+2}\right)}} 
                \leq \frac{1}{2^{n}}, \; \forall n \in \mathbb{N}
            \] 
            Επομένως $ 0 \leq \frac{1}{2^{n}} 
            \abs{\sin{\frac{\left(n^{2}+1\right)}{n+2}}} \leq \frac{1}{2^{n}}, 
            \; \forall n \in \mathbb{N}$ και $ \sum_{n=1}^{\infty} \frac{1}{2^{n}} = 
            \sum_{n=1}^{\infty} \left(\frac{1}{2}\right)^{n} $ συγκλίνει 
            ως γεωμετρική με $ \abs{\lambda} = \abs{\frac{1}{2}} = \frac{1}{2} < 1 $, 
            επομένως από κριτήριο Σύγκρισης και $ \sum_{n=1}^{\infty} \frac{1}{2^{n}} 
            \abs{\sin{\left(\frac{n^{2}+1}{n+2}\right)}} $ συγκλίνει.

        \item Παρατηρούμε ότι $ a_{n} = \frac{\abs{\sin{(n^{2}+1)} \cdot 
            \cos{(n+5)}}}{n^{4}+5} \geq 0, \; \forall n \in \mathbb{N}$. Έχουμε
            \[
                a_{n} = \frac{\abs{\sin{(n^{2}+1)} \cdot \cos{(n+5)}}}{n^{4}+5} = 
                \frac{\abs{\sin{(n^{2}+1)}} \cdot \abs{\cos{(n+5)}}}{n^{4}+5} \leq
                \frac{1 \cdot 1}{n^{4}+5} \leq \frac{1}{n^{4}}, \; \forall n 
                \in \mathbb{N}
            \] 
            Επομένως $ 0 \leq  \frac{\abs{\sin{(n^{2}+1)} \cdot 
            \cos{(n+5)}}}{n^{4}+5} \leq \frac{1}{n^{4}}, \; \forall n \in 
            \mathbb{N} $ και $ \sum_{n=1}^{\infty} \frac{1}{n^{4}} $ συγκλίνει 
            ως γενικευμένη αρμονική με $ \rho = 4 > 1 $, επομένως από 
            κριτήριο Σύγκρισης και $ \sum_{n=1}^{\infty} \frac{\abs{\sin{(n^{2}+1)}
            \cdot \cos{(n+5)}}}{n^{4}+5}  $ συγκλίνει.

        \item Παρατηρούμε ότι $ a_{n} = \frac{\sin^{4}{n}}{1+ \sqrt{n^{5}}} \geq 0, 
            \; \forall n \in \mathbb{N}$. Έχουμε
            \[
                a_{n} =  \frac{\sin^{4}{n}}{1+ \sqrt{n^{5}}} = \frac{(\sin{n} )^{4}}{1+
                \sqrt{n^{5}}} \leq \frac{1^{4}}{1+ \sqrt{n^{5}}} < 
                \frac{1}{\sqrt{n^{5}}} = \frac{1}{n^{\frac{5}{2}}}, \; 
                \forall n \in \mathbb{N}
            \] 
            Επομένως $ 0 \leq \frac{\sin^{4}{n}}{1 + \sqrt{n^{5}}} \leq
            \frac{1}{n^{\frac{5}{2}}}, \; \forall n \in \mathbb{N} $ και 
            $ \sum_{n=1}^{\infty} \frac{1}{n^{\frac{5}{2}}} $ συγκλίνει ως 
            γενικευμένη αρμονική με $ \rho = \frac{5}{2} > 1 $, επομένως και 
            $ \sum_{n=1}^{\infty} \frac{\sin^{4}{n}}{1 + \sqrt{n^{5}}} $ συγκλίνει.

        \item Παρατηρούμε ότι $ a_{n}= \frac{3^{n}+5}{4^{n}+n^{2}} \geq 0, 
            \; \forall n \in \mathbb{N}$. Έχουμε
            \[
                a_{n} = \frac{3^{n}+5}{4^{n}+n^{2}} \leq \frac{3^{n}+5}{4^{n}} = 
                \frac{3^{n}}{4^{n}} + \frac{5}{4^{n}} = 
                \left(\frac{3}{4}\right)^{n} + 5 \left(\frac{1}{4}\right)^{n} , \; 
                \forall n \in \mathbb{N}
            \] 
            Επομένως $ 0 \leq \frac{3^{n}+5}{4^{n}+n^{2}} \leq 
            \left(\frac{3}{4}\right)^{n} + 5 \left(\frac{1}{4}\right)^{n}, \; 
            \forall n \in \mathbb{N} $ και $ 
            \sum_{n=1}^{\infty} \left(\frac{3}{4} \right)^{n}  $ και $
            \sum_{n=1}^{\infty} 5 \left(\frac{1}{4} \right)^{n} $ 
            συγκλίνουν ως γεωμετρικές με 
            $ \abs{\lambda_{1}} = \abs{\frac{3}{4}} = \frac{3}{4} < 1 $ και 
            $ \abs{\lambda_{2}} = \abs{\frac{1}{4}} = \frac{1}{4} < 1 $ αντίστοιχα, 
            επομένως από κριτήριο Σύγκρισης και 
            $ \sum_{n=1}^{\infty} \frac{3^{n}+5}{4^{n}+n^{2}} $ 
            συγκλίνει ως το άθροισμα συγκλινουσών σειρών.

        \item Παρατηρούμε ότι $ a_{n} = \frac{1}{n!} \geq 0, \; \forall n \in 
            \mathbb{N} $. Έχουμε 
            \[
                a_{n}= \frac{1}{n!} = \frac{1}{1} \cdot \frac{1}{2} \cdot 
                \frac{1}{3} \cdots
                \frac{1}{n} = \frac{1}{2} \cdot \frac{1}{3} \cdots \frac{1}{n} \leq 
                \frac{1}{2} \cdot \frac{1}{2} \cdots \frac{1}{2} = \frac{1}{2^{n-1}} = 
                \left(\frac{1}{2} \right)^{n-1}, 
                \; \forall n \in \mathbb{N} 
            \] 
            Επομένως $ 0 \leq \frac{1}{n!} \leq \frac{1}{2^{n-1}}, \; \forall n \in 
            \mathbb{N} $ και $ \sum_{n=1}^{\infty} \left(\frac{1}{2} \right)^{n-1}  $ 
            συγκλίνει, ως γεωμετρική με $ \abs{\lambda} = \abs{\frac{1}{2}} = 
            \frac{1}{2} < 1 $, επομένως από κριτήριο Σύγκρισης και 
            $ \sum_{n=1}^{\infty} \frac{1}{n!} $ συγκλίνει.

        \item Παρατηρούμε ότι $ a_{n}= \frac{n!}{2^{n}+1} \geq 0, \; 
            \forall n \in \mathbb{N} $.  Έχουμε 
            \[
                a_{n} = \frac{n!}{2^{n}+1} \geq \frac{2^{n-1}}{2^{n}+1} \geq 
                \frac{2^{n-1}}{2^{n}+2^{n}} = 
                \frac{2^{n}\cdot 2^{-1}}{2\cdot 2^{n}} = \frac{1}{4}, 
                \; \forall n \in \mathbb{N}
            \] 
            Επομένως $ 0 \leq \frac{1}{4} \leq \frac{n!}{2^{n}+1}, \; 
            \forall n \in \mathbb{N}
            $ και $ \sum_{n=1}^{\infty} \frac{1}{4} $ αποκλίνει, επομένως από 
            κριτήριο Σύγκρισης και $ \sum_{n=1}^{\infty} \frac{n!}{2^{n}+1} $ 
            αποκλίνει.

        \item Παρατηρούμε ότι $ a_{n}= \frac{1}{n} \cdot \sin{\frac{\pi}{n}} \geq 0, 
            \; \forall n \in \mathbb{N}$. Έχουμε 
            \[
                a_{n} = \frac{1}{n} \cdot \sin{\frac{\pi}{n}} = \frac{1}{n} 
                \cdot \abs{\sin{\frac{\pi}{n}}} \leq \frac{1}{n} \cdot 
                \abs{\frac{\pi}{n}} = \frac{\pi}{n^{2}}, \; \forall n \in \mathbb{N} 
            \] 
            Επομένως $ 0 \leq \frac{1}{n} \cdot \sin{\frac{\pi}{n}} \leq 
            \frac{\pi}{n^{2}}, \; \forall n \in \mathbb{N}$ και $ \sum_{n=1}^{\infty} 
            \frac{\pi}{n^{2}} = \pi \sum_{n=1}^{\infty} \frac{1}{n^{2}}$ συγκλίνει, 
            επομένως από κριτήριο και σύγκλισης και $ \sum_{n=1}^{\infty} 
            \frac{1}{n} \cdot \sin{\frac{\pi}{n}}$ συγκλίνει.
    \end{enumerate} 

\item 
    \begin{enumerate}[i)]
        \item 
            \[
                \frac{\abs{a_{n+1}}}{\abs{a_{n}}} = 
                \frac{\frac{(n+1)!}{3^{n+1}}}{\frac{n!}{3^{n}}} = 
                \frac{3^{n}}{3^{n+1}} \cdot
                \frac{(n+1)!}{n!} = \frac{3^{n}}{3^{n}\cdot 3} \cdot \frac{1 \cdot 2 
                \cdots n \cdot (n+1))}{1 \cdot 2 \cdots n} = \frac{1}{3} \cdot (n+1) 
                \xrightarrow{n \to \infty} \infty > 1
            \] 
            Επομένως από κριτήριο Λόγου η σειρά $ \sum_{n=1}^{\infty} 
            \frac{n!}{3^{n}}$ αποκλίνει. 

        \item 
            \[
                \frac{\abs{a_{n+1}}}{\abs{a_{n}}} = \frac{\frac{2^{n+1}}{(n+1)^{2}}}{
                \frac{2^{n}}{n^{2}}} = \frac{n^{2}}{(n+1)^{2}} \cdot 
                \frac{2^{n+1}}{2^{n}} =  \left(\frac{n}{n+1}\right)^{2} \cdot 2 
                \xrightarrow{n \to \infty} 1^{2}\cdot 2 = 2 > 1 
            \] 
            Επομένως από κριτήριο Λόγου η σειρά $ \sum_{n=1}^{\infty} 
            \frac{2^{n}}{n^{2}} $ αποκλίνει.

        \item 
            \[
                \frac{\abs{a_{n+1}}}{\abs{a_{n}}} = \frac{\frac{3^{n+1}\cdot (n+1)!}
                {(n+1)^{n+1}}}{\frac{3^{n}\cdot n!}{n^{n}}} = \frac{3^{n+1}}{3^{n}} 
                \cdot \frac{n^{n}}{(n+1)^{n+1}} \cdot \frac{(n+1)!}{n!} = 3 \cdot 
                \frac{n^{n}}{(n+1)^{n+1}} \cdot (n+1) = 3 \cdot \left(\frac{n}{n+1} 
                \right)^{n} 
            \] 
            Όπου 
            \[
                \lim_{n \to \infty} \left(\frac{n}{n+1} \right)^{n} = 
                \lim_{n \to \infty}\frac{1}{\left(\frac{n+1}{n}\right)^{n}} = 
                \lim_{n \to \infty} \frac{1}{(1 + \frac{1}{n})^{n}} = \frac{1}{e}
            \]
            Επομένως $ \lim_{n \to \infty} \frac{\abs{a_{n+1}}}{\abs{a_{n}}} = 
            3 \cdot \frac{1}{e} = \frac{3}{e} > 1$ άρα από κριτήριο Λόγου η σειρά 
            $ \sum_{n=1}^{\infty} \frac{3^{n}\cdot n!}{n^{n}} $ αποκλίνει.

        \item 
            \[
                \frac{\abs{a_{n+1}}}{\abs{a_{n}}} = 
                \frac{\frac{1}{e^{n+1}}}{\frac{1}{e^{n}}} =
                \frac{e^{n}}{e^{n+1}} = \frac{e^{n}}{e\cdot e^{n}} = \frac{1}{e} <1  
            \] 
            Επομένως από κριτήριο Λόγου η σειρά $ \sum_{n=1}^{\infty} \frac{1}{e^{n}} $
            συγκλίνει.

        \item 
            \[
                \frac{\abs{a_{n+1}}}{\abs{a_{n}}} = \frac{\frac{6^{n+1}}{2(n+1)+7}}
                {\frac{6^{n}}{2n+7}} = \frac{6^{n}\cdot 6}{6^{n}} \cdot 
                \frac{2n+7}{2n+9} = 6 \cdot \frac{2n+7}{2n+9} 
                \xrightarrow{n \to \infty} 6 \cdot 1 = 6>1
            \] 
            Επομένως από κριτήριο Λόγου η σειρά 
            $ \sum_{n=1}^{\infty} \frac{6^{n}}{2n+7} $ αποκλίνει.

        \item 
            \begin{align*}
                \frac{\abs{a_{n+1}}}{\abs{a_{n}}}
                &=\frac{\frac{3^{n+1}+4^{n+1}}{5^{n+1}
                +4^{n+1}}}{\frac{3^{n}+4^{n}}{5^{n}+4^{n}}} = 
                \frac{3 \cdot 3^{n}+ 4\cdot 4^{n}}{3^{n}+4^{n}} \cdot 
                \frac{5^{n}+4^{n}}{5\cdot 5^{n}+ 4 \cdot 4^{n}} = \frac{4^{n}
                    \left(3\cdot (\frac{3}{4})^{n}+4 
                \cdot 1\right)}{4^{n}\left((\frac{3}{4} )^{n}+ 1\right)} \cdot 
                \frac{5^{n}\left(1+ (\frac{4}{5})^{n}\right)}{5^{n}\left(5\cdot 1 + 4
                \cdot(\frac{4}{5} )^{n}\right)} \\ 
                &= \frac{3\cdot (\frac{3}{4})^{n}+4}{(\frac{3}{4} )^{n}+ 1} \cdot 
                \frac{1+ (\frac{4}{5})^{n}}{5 + 4 \cdot(\frac{4}{5} )^{n}} 
                \xrightarrow{n \to \infty} \frac{3 \cdot 0 + 4}{0 + 1 } \cdot 
                \frac{1 + 0}{5 + 4 \cdot 0} = \frac{4}{5} <1
            \end{align*} 
            Επομένως από κριτήριο Λόγου η σειρά $ \sum_{n=1}^{\infty}
            \frac{3^{n}+4^{n}}{5^{n}+4^{n}} $ συγκλίνει.

        \item 
            \[
                \frac{\abs{a_{n+1}}}{\abs{a_{n}}} =
                \frac{\frac{[(n+1)!]^{2}}{[2(n+1)]!}}{\frac{(n!)^{2}}{(2n)!}} = 
                \left(\frac{(n+1)!}{n!}\right)^{2} \cdot \frac{(2n)!}{(2n+2)!} = 
                \frac{(n+1)^{2} }{(2n+1)\cdot (2n+2)} = 
                \frac{n^{2}+2n+1}{4n^{2}+6n+2} \xrightarrow{n \to \infty} 
                \frac{1}{4} <1
            \] 
            Επομένως από κριτήριο Λόγου η σειρά $ \sum_{n=1}^{\infty} \frac{(n!)^{2}}
            {(2n)!} $ συγκλίνει.

        \item 
            \[
                \frac{\abs{a_{n+1}}}{\abs{a_{n}}} = \frac{\frac{(n+1)!}{1\cdot 3 \cdots
                (2(n+1)-1)}}{\frac{n!}{1\cdot 3 \cdots (2n-1)}} = 
                \frac{(n+1)!}{n!} \cdot \frac{1\cdot 3 \cdots (2n-1)}{1\cdot 3 \cdots 
                (2n-1)\cdot (2n+1)} = \frac{n+1}{2n+1} \xrightarrow{n \to \infty} 
                \frac{1}{2} <1 
            \] 
            Επομένως από κριτήριο Λόγου η σειρά $ \sum_{n=1}^{\infty} 
            \frac{n!}{1\cdot 3 \cdots (2n-1)}  $ συγκλίνει.
    \end{enumerate}

\item 
    \begin{enumerate}[i)]
        \item 
            \[
                \sqrt[n]{\abs{a_{n}}} = \sqrt[n]{\left(\frac{3n}{n+1} \right)^{n}} = 
                \frac{3n}{n+1}\xrightarrow{n \to \infty} 3 > 1 
            \]
            Επομένως από κριτήριο Ρίζας η σειρά $ \sum_{n=1}^{\infty} 
            \left(\frac{3n}{n+1}\right)^{n} $ αποκλίνει.

        \item 
            \[
                \sqrt[n]{\abs{a_{n}}} = \sqrt[n]{\frac{2^{n}}{n^{n}}} = 
                \sqrt[n]{\left(\frac{2}{n} \right)^{n}} = \frac{2}{n} 
                \xrightarrow{n \to \infty} 0 < 1
            \] 
            Επομένως από κριτήριο Ρίζας η σειρά $ \sum_{n=1}^{\infty} 
            \frac{2^{n}}{n^{n}} $ συγκλίνει.

        \item 
            \[
                \sqrt[n]{\abs{a_{n}}} = \sqrt[n]{\frac{e^{n}}{5n}} = 
                \frac{e}{\sqrt[n]{5} \cdot \sqrt[n]{n}} \xrightarrow{n \to \infty} 
                \frac{e}{1 \cdot 1} = e>1
            \] 
            Επομένως από κριτήριο Ρίζας η σειρά 
            $ \sum_{n=1}^{\infty} \frac{e^{n}}{5n} $ αποκλίνει.

        \item 
            \[
                \sqrt[n]{\abs{a_{n}}} = \sqrt[n]{\frac{n^{3}}{e^{n^{2}}}} = 
                \frac{\sqrt[n]{n^{3}} }{e^{n}} = \frac{\sqrt[n]{n} \cdot 
                \sqrt[n]{n} \cdot \sqrt[n]{n}}{e^{n}} \xrightarrow{n \to \infty} 
                1\cdot 0 = 0 <1
            \] 
            Επομένως από κριτήριο Ρίζας η σειρά 
            $ \sum_{n=1}^{\infty} \frac{n^{3}}{e^{n^{2}}} $ συγκλίνει.

        \item 
            \[
                \sqrt[n]{\abs{a_{n}}} = \sqrt[n]{\frac{2^{n}}{n\cdot e^{n+1}}} = 
                \frac{2}{\sqrt[n]{n} \cdot \sqrt[n]{e^{n+1}}} = 
                \frac{2}{\sqrt[n]{n} \cdot \sqrt[n]{e^{n}}\cdot \sqrt[n]{e}}
                \xrightarrow{n \to \infty} \frac{2}{1 \cdot e \cdot 1} = 
                \frac{2}{e} <1
            \] 
            Επομένως από κριτήριο Ρίζας η σειρά 
            $ \sum_{n=1}^{\infty} \frac{2^{n}}{n \cdot e^{n+1}} $ συγκλίνει.

        \item 
            \[
                \sqrt[n]{\abs{a_{n}}} = \sqrt[n]{(\sqrt[n]{n} -1)^{n}} = 
                \sqrt[n]{n} - 1 \xrightarrow{n \to \infty} 1-1=0 <1
            \]
            Επομένως από κριτήριο Ρίζας η σειρά 
            $ \sum_{n=1}^{\infty} (\sqrt[n]{n} -1)^{n} $ συγκλίνει.

        \item 
            \[
                \sqrt[n]{\abs{a_{n}}} = \sqrt[n]{\frac{1}{3^{n} 
                \cdot (\frac{n+1}{n})^{n^{2}}}} = \sqrt[n]{\frac{1}{3^{n}}} 
                \cdot \sqrt[n]{\left(\frac{n+1}{n} \right)^{n^{2}}} = 
                \frac{1}{3} \cdot \left(\frac{n+1}{n} \right)^{n} = \frac{1}{3} 
                \cdot \left(1+ \frac{1}{n} \right)^{n} \xrightarrow{n \to \infty} 
                \frac{1}{3} \cdot e = \frac{e}{3} <1 
            \] 
            Επομένως από κριτήριο Ρίζας η σειρά 
            $ \sum_{n=1}^{\infty} \frac{1}{3^{n}} \cdot 
            \left(\frac{n+1}{n} \right)^{n^{2}} $ συγκλίνει.

        \item 
            \[
                \sqrt[n]{\abs{a_{n}}} = \sqrt[n]{\left(1+ \frac{1}{4n} 
                \right)^{-n^{2}}} = \left(1+ \frac{1}{4n} \right)^{-n} = 
                \frac{1}{\left(1+ \frac{1}{4n} \right)^{n}} = 
                \frac{1}{\left(1 + \frac{\frac{1}{4}}{n}\right)^{n}} 
                \xrightarrow{n \to \infty} \frac{1}{e ^{\frac{1}{4}}} = 
                \frac{1}{\sqrt[4]{e}} <1 
             \] 
             Επομένως από κριτήριο Ρίζας η σειρά 
             $ \sum_{n=1}^{\infty} \left(1+ \frac{1}{4n} \right)^{-n^{2}} $ συγκλίνει.
    \end{enumerate}

\item 
    \begin{enumerate}[i)]
        \item Παρατηρούμε ότι $ a_{n}= \frac{1}{n^{2}+n-1} > 0, 
            \; \forall n \in \mathbb{N}$. Θέτουμε 
            $
            b_{n} = \frac{1}{n^{2}} > 0, \; \forall n \in \mathbb{N} 
            $ οπότε
            \[
                \frac{a_{n}}{b_{n}} = \frac{\frac{1}{n^{2}+n-1}}{\frac{1}{n^{2}}} = 
                \frac{n^{2}}{n^{2}+n-1} \xrightarrow{n \to \infty} 1 \in \mathbb{R} 
            \] 
            και επειδή $ \sum_{n=1}^{\infty} b_{n} = \sum_{n=1}^{\infty} 
            \frac{1}{n^{2}} $ συγκλίνει, τότε από κριτήριο Ορίου και η σειρά 
            $ \sum_{n=1}^{\infty} a_{n} $ συγκλίνει.

        \item Παρατηρούμε ότι $ a_{n}= \frac{n}{3n^{2}-4} $ δεν είναι ακολουθία 
            θετικών όρων γιατί για $ n=1 $ έχουμε $ a_{1} = -1 $. Οπότε δεν 
            εφαρμόζεται το κριτήριο Ορίου. Γι᾽ αυτό 
            \[
                a_{n}= \frac{n}{3n^{2}-4} \geq \frac{n}{3n^{2}} = \frac{1}{3n} =b_{n}, 
                \; \forall n \in \mathbb{N}
            \] 
            Όμως 
            \[
                0 \leq \frac{1}{3n} \leq \frac{n}{3n^{2}-4}, \; \forall n \geq 2 
            \]
            και επειδή η σειρά $ \sum_{n=1}^{\infty} \frac{1}{3n} = 
            \sum_{n=1}^{\infty} \frac{1}{3} \cdot \frac{1}{n}$ αποκλίνει, 
            τότε από κριτήριο Σύγκρισης και 
            $ \sum_{n=1}^{\infty} \frac{n}{3n^{2}-4}$ αποκλίνει.

        \item Παρατηρούμε ότι $ a_{n}= \frac{n^{3}+2n+1}{5n^{5}-7} $ όχι 
            ακολουθία θετικών όρων, αφού για $ n=1 $, έχουμε $ a_{1}=-4$. Οπότε δεν 
            εφαρμόζεται το κριτήριο Ορίου. Γι᾽ αυτό
            \[
                0 < a_{n}= \frac{n^{3}+2n+1}{5n^{5}-7} < 
                \frac{n^{3}+2n^{3}+n^{3}}{5n^{5}-7} < \frac{4n^{3}}{5n^{5}-n^{3}} = 
                \frac{4n^{3}}{n^{3}(5n^{2}-1)} = \underbrace{\frac{4}{5n^{2}-1}}_
                {b_{n}}, \; \forall n \geq 2
            \] 
            και επειδή για τη σειρά $ \sum_{n=1}^{\infty} \frac{4}{5n^{2}-1} $ 
            έχουμε ότι συγκλίνει, γιατί 
            $
                b_{n} = \frac{4}{5n^{2}-1} \geq 0, \; \forall n \in \mathbb{N}
            $ 
            και $ c_{n} = \frac{1}{n^{2}} $, τότε
            \[
                \frac{b_{n}}{c_{n}} = \frac{\frac{4}{5n^{2}-1}}{\frac{1}{n^{2}}} = 
                \frac{4n^{2}}{5n^{2}-1} \xrightarrow{n \to \infty} \frac{4}{5} 
                \in \mathbb{R} 
            \]
            οπότε από κριτήριο Ορίου η σειρά $ \sum_{n=1}^{\infty} b_{n}= 
            \sum_{n=1}^{\infty} \frac{4}{5n^{2}-1}$ συγκλίνει και τελικά 
            από το κριτήριο Σύγκρισης
            και η σειρά $ \sum_{n=1}^{\infty} a_{n} = \sum_{n=1}^{\infty}
            \frac{n^{3}+2n+1}{5n^{5}-7} $ θα συγκλίνει.

        \item Παρατηρούμε ότι $ a_{n}= \frac{1}{\sqrt{n^{3}+n^{2}}} \geq 0, 
            \; \forall n \in \mathbb{N}$. Θέτουμε $ b_{n} = \frac{1}{n^{\frac{3}{2}}} $.
            Οπότε
            \[
                \frac{a_{n}}{b_{n}} =
                \frac{\frac{1}{\sqrt{n^{3}+n^{2}}}}{\frac{1}{n^{\frac{3}{2}}}} = 
                \frac{\sqrt{n^{3}}}{\sqrt{n^{3}+n^{2}}} = 
                \sqrt{\frac{n^{3}}{n^{3}+n^{2}}}
                \xrightarrow{n \to \infty} \sqrt{1}=1 \in \mathbb{R} 
             \] 
             και επειδή η η σειρά $ \sum_{n=1}^{\infty} \frac{1}{n^{\frac{3}{2}}} $ 
             συγκλίνει, ως γενικευμένη αρμονική με $ \rho = \frac{3}{2} >1 $, τότε 
             από κριτήριο Ορίου και η σειρά $ \sum_{n=1}^{\infty} 
             \frac{1}{\sqrt{n^{3}+n^{2}} } $ συγκλίνει.

     \item Παρατηρούμε ότι $ a_{n}= \frac{n}{\sqrt{n^{3}+n^{2}}} \geq 0, 
         \; \forall n \in \mathbb{N} $. Θέτουμε $ b_{n}= \frac{1}{n^{\frac{1}{2}}} $. 
         Οπότε
         \[
             \frac{a_{n}}{b_{n}} = 
             \frac{\frac{n}{\sqrt{n^{3}+n^{2}}}}{\frac{1}{n^{\frac{1}{2}}}} =
             \frac{\sqrt{n}\cdot n}{\sqrt{n^{3}+n^{2}}} = 
             \sqrt{\frac{n^{3}}{n^{3}+n^{2}}} 
             \xrightarrow{n \to \infty} \sqrt{1}=1 \in \mathbb{R}
          \] 
          και επειδή η σειρά $ \sum_{n=1}^{\infty} \frac{1}{n^{\frac{1}{2}}} $ 
          αποκλίνει, ως γενικευμένη αρμονική με $ \rho = \frac{1}{2} <1 $, 
          τότε από κριτήριο Ορίου θα αποκλίνει και η σειρά $ 
          \sum_{n=1}^{\infty} \frac{n}{\sqrt{n^{3}+n^{2}}}$.
    \end{enumerate}

\item 
    \begin{enumerate}[i)]
        \item Παρατηρούμε ότι $ a_{n}= \frac{1}{n} \geq 0, \; \forall n \in \mathbb{N}$.
            Έχουμε
            \begin{myitemize}
            \item $ n+1>n, \; \forall n \in \mathbb{N} \Leftrightarrow \frac{1}{n+1} < 
                \frac{1}{n}, \; \forall n \in \mathbb{N} \Leftrightarrow a_{n+1} < 
                a_{n}, \; \forall n \in \mathbb{N} $ 
                επομένως $ a_{n} = \frac{1}{n} $ είναι
                γνησίως φθίνουσα και επομένως φθίνουσα.
            \item $ \lim_{n \to \infty} a_{n} = \lim_{n \to \infty} \frac{1}{n} = 0 $
            \end{myitemize}
            Επομένως από Κριτήριο Leibnitz η εναλλάσσουσα σειρά $ \sum_{n=1}^{\infty} 
            (-1)^{n+1} \frac{1}{n}$ συγκλίνει.

        \item Παρατηρούμε ότι $ a_{n}= \frac{1}{2n+1} \geq 0, 
            \; \forall n \in \mathbb{N}$. Έχουμε
            \begin{myitemize}
            \item $ n+1 > n, \; \forall n \in \mathbb{N} \Leftrightarrow 
                2(n+1) > 2n, \; \forall n \in \mathbb{N} \Leftrightarrow 
                2(n+1)+1 > 2n +1, \; \forall n \in \mathbb{N} \Leftrightarrow 
                \frac{1}{2(n+1)+1} < \frac{1}{2n+1}, \; \forall n \in \mathbb{N} 
                \Leftrightarrow a_{n+1} < a_{n}, \; \forall n \in \mathbb{N} $, 
                επομένως η $ a_{n}= \frac{1}{2n+1} $ είναι γνησίως φθίνουσα και άρα
                φθίνουσα.
            \item $ \lim_{n \to \infty} \frac{1}{2n+1} = \lim_{n \to \infty} \frac{1}
                {n(2+\frac{1}{n})} = \lim_{n \to \infty} 
                \left(\frac{1}{n} \cdot \frac{1}{2+ \frac{1}{n}}\right) = 
                \lim_{n \to \infty} \frac{1}{n} \cdot \lim_{n \to \infty} \frac{1}{2+
                \frac{1}{n}} = 0 \cdot \frac{1}{2+0} = 0$
            \end{myitemize}
            Επομένως από Κριτήριο Leibnitz η εναλλάσσουσα σειρά 
            $ \sum_{n=1}^{\infty} (-1)^{n} \frac{1}{2n+1} $ συγκλίνει.

        \item Παρατηρούμε ότι $ a_{n}= \frac{1}{\sqrt{n}} \geq 0, 
            \; \forall n \in \mathbb{N}$. Έχουμε
            \begin{myitemize}
            \item $ n+1>n, \; \forall n \in \mathbb{N} \Leftrightarrow 
                \sqrt{(n+1)} > \sqrt{n}, \; \forall n \in \mathbb{N} \Leftrightarrow 
                \frac{1}{\sqrt{(n+1)}} < \frac{1}{\sqrt{n}}, \; \forall n \in \mathbb{N}
                \Leftrightarrow a_{n+1} < a_{n}, \; \forall n \in \mathbb{N}$, άρα η 
                $ a_{n}= \frac{1}{\sqrt{n}} $ είναι γνησίως φθίνουσα και άρα φθίνουσα.
            \item $ \lim_{n \to \infty} \frac{1}{\sqrt{n}} = 0 $
            \end{myitemize}
            Επομένως από κριτήριο Leibnitz η εναλλάσσουσα σειρά $ \sum_{n=1}^{\infty} 
            (-1)^{n-1} \frac{1}{\sqrt{n}}$ συγκλίνει.

        \item Ομοίως με την προηγούμενη.

        \item $ \sum_{n=1}^{\infty} \left(- \frac{1}{2}\right)^{n} = 
            \sum_{n=1}^{\infty} (-1)^{n} 
            \left(\frac{1}{2}\right)^{n} $ και πρόκειται για εναλλάσσουσα σειρά με 
            $ a_{n}= \left(\frac{1}{2}\right)^{n} \geq 0, \; \forall n \in \mathbb{N} $.
            \begin{myitemize}
            \item $ \frac{a_{n+1}}{a_{n}} = 
                \frac{(\frac{1}{2} )^{n+1}}{(\frac{1}{2} )^{n}} =
                \frac{1}{2} < 1, \; \forall n \in \mathbb{N}  $, επομένως η $ 
                a_{n}= \left(\frac{1}{2} \right)^{n}$ είναι γνησίως φθίνουσα 
                και επομένως φθίνουσα.
                $ \lim_{n \to \infty} \left(\frac{1}{2} \right)^{n} = 0 $
            \end{myitemize}
            Επομένως από κριτήριο Leibnitz η εναλλάσσουσα σειρά $ \sum_{n=1}^{\infty} 
            (-1)^{n}\left(\frac{1}{2}\right)^{n} $ συγκλίνει.
    \end{enumerate}

    \begin{rem}
        Για αυτήν την άσκηση θα μπορούσε να είχε χρησιμοποιηθεί και το 
        πιο γενικό κριτήριο Dirichlet, όπου μία ακόμη προϋπόθεση για την ακολουϑία 
        $ b_{n} = (-1)^{n}$ είναι να έχει φραγμένα μερικά αθροίσματα, δηλαδή 
        \[ \exists M>0 \; : \; \abs{\sum_{n=1}^{N} (-1)^{n}} < M, \; 
        \forall n \in \mathbb{N}.\] Πράγματι, έχουμε
        \[
            \abs{\sum_{n=1}^{N} (-1)^{n}} = 0 \; \text{ή} \; 1 \leq 1=M, \; 
            \forall n \in \mathbb{N}  
         \] 
         οπότε ισχύουν οι προϋποθέσεις του κριτηρίου Dirichlet και επομένως όλες 
         οι σειρές της παραπάνω ασκήσεις $ \sum_{n=1}^{\infty} b_{n} \cdot a_{n} $ 
         συγκλίνουν, όπου $ b_{n} = (-1)^{n} $ ή $ b_{n} = \cos{n \pi} $.
    \end{rem}

\item 
    \begin{enumerate}[i)]
        \item 
            \[
                \abs{a_{n}} = \abs{(-1)^{n-1} \frac{ne}{n!}} = \frac{ne}{n!}
             \] 
             \[
                 \frac{\abs{a_{n+1}}}{\abs{a_{n}}} = \frac{\frac{(n+1)e}{(n+1)!}}
                 {\frac{ne}{n!}} = \frac{(n+1)e}{ne} \cdot \frac{n!}{(n+1)!} = 
                 \frac{n+1}{n} \cdot \frac{1}{n+1} = \frac{1}{n} = 0<1
              \] 
              Επομένως από κριτήριο Λόγου η σειρά $ \sum_{n=1}^{\infty} \abs{a_{n}} $ 
              συγκλίνει και επομένως από απόλυτη σύγκλιση η σειρά 
              $ \sum_{n=1}^{\infty} a_{n} $ συγκλίνει.

          \item 
              \[
                  \abs{a_{n}} = \abs{(-1)^{n-1} \frac{n^{2}}{n^{4}+2}} = 
                  \frac{n^{2}}{n^{4}+2}  
               \] 
               η οποία εύκολα με κριτήριο σύγκρισης ή κριτήριο Ορίου αποδεικνύεται ότι 
               συγκλίνει και άρα η σειρά $ \sum_{n=1}^{\infty} \abs{a_{n}} $ συγκλίνει
               και από απόλυτη σύγκλιση η σειρά $ \sum_{n=1}^{\infty} a_{n} $ συγκλίνει.

           \item 
               \[
                   \abs{a_{n}} = \abs{(-1)^{n+1} \frac{\cos{n} + \sin{n}}{5^{n}}} = 
                   \abs{\frac{\cos{n} + \sin{n}}{5^{n}}} = 
                   \frac{\abs{\cos{n} + \sin{n}}}{5^{n}}
                   \leq \frac{\abs{\cos{n}} + \abs{\sin{n}}}{5^{n}} 
                   \leq \frac{1+1}{5^{n}} = \frac{2}{5^{n}}  
                \] 
                όπου η σειρά $ \sum_{n=1}^{\infty} \abs{a_{n}} = \sum_{n=1}^{\infty} 
                \frac{2}{5^{n}} = 2 \sum_{n=1}^{\infty} \left(\frac{1}{5} \right)^{n} $
                η οποία συγκλίνει ως γεωμετρική με 
                $ \abs{\lambda} = \abs{\frac{1}{5}} = \frac{1}{ 5} <1 $ και άρα από 
                απόλυτη σύγκλιση και η σειρά $ \sum_{n=1}^{\infty} a_{n} $ συγκλίνει.

            \item 
                \[
                    \abs{a_{n}} = \abs{\frac{1}{2^{n}} \cdot 
                    \sin{\left(\frac{n^{2}+1}{n+2}\right)}} =
                    \frac{1}{2^{n}} \cdot \abs{\sin{\left(\frac{n^{2}+1}{n+2}\right)}} 
                    \leq \frac{1}{2^{n}} \cdot 1 = \frac{1}{2^{n}}
                 \] 
                 όπου η σειρά $ \sum_{n=1}^{\infty} \abs{a_{n}} = 
                 \sum_{n=1}^{\infty} \frac{1}{2^{n}}  $ εύκολα αποδεικνύεται με 
                 κριτήριο λόγου ότι συγκλίνει και άρα από απόλυτη σύγκλιση και η σειρά
                 $ \sum_{n=1}^{\infty} a_{n} $ συγκλίνει.
    \end{enumerate}
\end{enumerate}



\end{document}

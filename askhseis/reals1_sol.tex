\input{preamble_twocol.tex}
\input{definitions.tex}



\setlength{\columnseprule}{1pt}
\setlength{\columnsep}{20pt}
\def\columnseprulecolor{\color{Col1}}

\renewcommand{\qedsymbol}{}


\begin{document}
\pagestyle{vangelis}

\begin{center}
  \large\bfseries\textcolor{Col2}{Πραγματικοί Αριθμοί} \\
  \large\bfseries\textcolor{Col2}{(Λύσεις των Ασκήσεων)}
\end{center}

\vspace{\baselineskip}


\begin{multicols}{2}

  \begin{enumerate}
    \setcounter{enumi}{1}
  \item \textcolor{Col1}{Για τα παρακάτω σύνολα, να υπολογιστούν και να αποδειχθούν τα 
    supremum και infimum.}
    \begin{enumerate}
      \item {\boldmath\textcolor{Col2}{$A = (0,2) $}}
        \begin{proof}(Με Ορισμό)
        \item {}
          Έχουμε ότι $ \sup A = 2 $ και $ \inf A = 0 $.
          \begin{myitemize}
            \item Θα δείξουμε ότι $ \sup A = 2 $. Πράγματι:
              $2$ α.φ. του $A$, γιατί $ a < 2, \; \forall a \in A $.
              Έστω $M$ α.φ. του $A$, δηλ. $a \leq M, \; \forall a \in A $.
              Θ.δ.ο. $ 2 \leq M $ (Με άτοπο). Πράγματι:
              Έστω $ M < 2 $. Τότε $ (M,2) \neq \emptyset \Rightarrow (M,2) \cap A \neq
              \emptyset $, άρα 
              $ \exists a \in (M,2) $.
              Δηλαδή $ \exists a \in A $ με $ a > M $. 
              άτοπο, γιατί $ M $ α.φ. του $A$.
            \item Θα δείξουμε ότι $ \inf A = 0 $. Πράγματι:
              $0$ κ.φ. του $A$, γιατί $ a > 0, \; \forall a \in A $.
              Έστω $m$ κ.φ. του $A$, δηλ. $a \geq m, \; \forall a \in A $.
              Θ.δ.ο. $ 0 \geq m $ (Με άτοπο). Πράγματι:
              Έστω $ m > 0 $. Τότε $ (0,m) \neq \emptyset \Rightarrow (0,M) \cap A \neq
              \emptyset $, άρα 
              $ \exists a \in (0,m) $. 
              Δηλαδή $ \exists a \in A $ με $ a < m $, 
              άτοπο, γιατί $ m $ κ.φ. του $A$.
          \end{myitemize}
        \end{proof}
      \item {\boldmath\textcolor{Col2}{$ A = \{ x \in \mathbb{R} \; 
        : \; x < 0 \} $}}. 
        \begin{proof}(Με Ορισμό)
        \item {}
          Έχουμε ότι $ \sup A = 0 $ και $ \inf A = - \infty $.
          \begin{myitemize}
            \item Θα δείξουμε ότι $ \sup A = 0 $. Πράγματι:
              $ 0 $ α.φ. του $A$, γιατί $ a < 0, \; \forall a \in A $.
              Έστω $M$ α.φ. του $A$, δηλ. $a \leq M, \; \forall a \in A $.
              Θ.δ.ο. $ 0 \leq M $ (Με άτοπο). Πράγματι:
              Έστω $ M < 0 $. Τότε $ (M,0) \neq \emptyset $, άρα 
              $ \exists a \in (M,0) $.
              Δηλαδή $ \exists a \in A $ με $ a > M $, 
              άτοπο, γιατί $ M $ α.φ. του $A$.
            \item Θα δείξουμε ότι $ \inf A = -\infty $. Πράγματι:
              $ A \neq \emptyset $ και $A$ όχι κάτω φραγμένο. Άρα γράφουμε
              $ \inf A = - \infty $.
          \end{myitemize}
        \end{proof}
      \item {\boldmath \textcolor{Col2}{$ A = \left\{ \frac{1}{n} \; 
        : \; n \in \mathbb{N} \right\} $}}.
        \begin{proof}
        \item {}
          Έχουμε ότι $ \sup A = 1 $ και $ \inf A = 0 $.
          \begin{myitemize}
            \item Θα δείξουμε ότι $ \sup A = 1 $. Πράγματι:
              $ \frac{1}{n} \leq 1, \; \forall n \in \mathbb{N} $, άρα το 1 είναι 
              α.φ. του $A$ και επίσης $ 1 \in A $ (για $ n=1 $). 
              Άρα $ 1 = \max A $. Άρα $ \sup A = \max A = 1 $.
            \item Θα δείξουμε ότι $ \inf A = 0 $. Πράγματι:
              Προφανώς $ 0 \leq \frac{1}{n}, \; \forall n \in \mathbb{N} $, 
              άρα το $ 0 $ κ.φ. του $A$.
              Έστω $ \varepsilon > 0 $. Τότε $ \exists n_{0} \in \mathbb{N} $
              με $ \frac{1}{n_{0}} < \varepsilon $ τ.ω. $ \frac{1}{n_{0}} 
              < 0 + \varepsilon$, με $ \frac{1}{n_{0}} \in A $.
          \end{myitemize}
        \end{proof}
      \item {\boldmath \textcolor{Col2}{$ A = \left\{ \frac{n}{n+1} \; 
        : \; n \in \mathbb{N} \right\} $ }}
        \begin{proof}
        \item {}
          Έχουμε ότι $ \sup A = 1 $ και $ \inf A = \frac{1}{2} $
          \begin{myitemize}
            \item Θα δείξουμε οτι $ \sup A = 1 $. Πράγματι:
              $ 1 $ α.φ. του $A$, γιατί $ \frac{n}{n+1} < 1, \; \forall n \in 
              \mathbb{N}$.
              (Δοκιμή: $ 1- \varepsilon < \frac{n}{n+1} \Leftrightarrow 
              \varepsilon > 1 - \frac{n}{n+1} \Leftrightarrow \frac{1}{n+1} 
              < \varepsilon \Leftrightarrow n+1 > \frac{1}{\varepsilon} 
              \Leftrightarrow n > \frac{1}{\varepsilon} -1$ )
              Έστω $ \varepsilon > 0 $. Τότε $ \exists n_{0} \in \mathbb{N}  $ με 
              $ n_{0} > \frac{1}{\varepsilon} - 1 $ τ.ω. $ 1 - \varepsilon < 
              \frac{n_{0}}{n_{0}+1}$, με $ \frac{n_{0}}{n_{0} +1} \in A $.
            \item Θα δείξουμε ότι $ \inf A = \frac{1}{2} $. Πράγματι: $ \frac{1}{2}
              $ κ.φ. του $A$, γιατί 
              $ \frac{n}{n+1} \geq \frac{1}{2} \Leftrightarrow n+1 \leq 2n \Leftrightarrow
              n \geq 1, \; \forall n \in \mathbb{N} $ (ισχύει)
              και $ \frac{1}{2} \in A$ (για $ n = 1 $).
              Άρα $ \frac{1}{2} = \min A $. Άρα $ \inf A = \min A = \frac{1}{2} $.
          \end{myitemize}
        \end{proof}
      \item {\boldmath \textcolor{Col2}{$ A = \left\{ \frac{1}{n} + (-1)^{n} \; 
        : \; n \in \mathbb{N} \right\} $}}.
        \begin{proof}
        \item {}
          Παρατηρούμε ότι \begin{align*} 
            A &= \left\{ 0, \frac{1}{2} + 1, \frac{1}{3} -1, 
            \frac{1}{4} + 1, \frac{1}{5} -1, \ldots  \right\} \\ 
              &= \left\{ 0, \frac{1}{3} -1, \frac{1}{5} -1, \ldots \right\} \cup 
              \left\{ \frac{1}{2} + 1, \frac{1}{4} + 1, \ldots \right\} \\ 
              &= \left\{ \frac{1}{2n-1} - 1 \; : \; n \in \mathbb{N}\right\}
              \cup \left\{ \frac{1}{2n} + 1 \; : \; n \in \mathbb{N}\right\} \\
              &= A_{1} \cup A_{2} 
            \end{align*}
            Αν παραστήσουμε τα στοιχεία του 
            $A=\{ 0, \frac{3}{2}, -\frac{2}{3}, \frac{5}{4}, -\frac{4}{5}, \ldots \}$, 
            πάνω στην ευθεία των πραγματικών αριθμών, παρατηρούμε ότι 
            $ \inf A = \inf A_{1} = -1 $ και $ \sup A = \sup A_{2} = \max A_{2} = 
            1+ \frac{1}{2} = \frac{3}{2} $. Πράγματι, από γνωστή πρόταση έχουμε ότι
            $ \sup A = \max \{ \sup A_{1}, \sup A_{2}\} = 
            \max \{0,\frac{3}{2}\} = \frac{3}{2}  $ και 
            $ \inf A = \min \{ \inf A_{1}, \inf A_{2} \} = \min \{ -1, 1 \} = -1 $ 
            Οπότε αρκεί να αποδείξουμε τα sup και inf των $A_{1}, A_{2} $. 
            \begin{myitemize}
              \item Θα δείξουμε ότι $ \sup A_{2} = \frac{3}{2} $. Πράγματι:
                $ \frac{3}{2} $ α.φ. του $A_{2} $, γιατί $ \frac{1}{2n}+1 \leq
                \frac{3}{2} \Leftrightarrow \frac{1+2n}{2n} \leq \frac{3}{2}
                \Leftrightarrow 2+4n \leq 6n \Leftrightarrow n \geq 1, 
                \; \forall n \in \mathbb{N} $ (ισχύει) 
                και $ \frac{3}{2} \in A_{2} $ (για $ n=1 $).
                Άρα $ \frac{3}{2} = \max A_{2} $. Άρα $ \sup A_{2} = \max A_{2} 
                = \frac{3}{2} $.
              \item Θα δείξουμε ότι $ \inf A_{1} = -1 $. Πράγματι:
                $ -1 $ κ.φ. του $A_{1}$, γιατί προφανώς $ \frac{1}{2n-1} - 1 \geq -1, 
                \; \forall n \in \mathbb{N} $.
                (Δοκιμή: $ \frac{1}{2 n -1} -1 < -1 + \varepsilon 
                \Leftrightarrow \frac{1}{2 n -1} < \varepsilon \Leftrightarrow 
                2n-1 > \frac{1}{\varepsilon} \Leftrightarrow n > \frac{1/ \varepsilon +
                1}{2}). $
                Έστω $ \varepsilon > 0 $. Τότε $ \exists n_{0} \in \mathbb{N}  $ 
                με $ n_{0} > \frac{1/ \varepsilon +1}{2} $ τ.ω. $ \frac{1}{2 n_{0} -1} 
                - 1 < \varepsilon -1 $, με $ \frac{1}{2 n_{0}-1} - 1 \in A $.
            \end{myitemize}
          \end{proof}
        \item {\boldmath \textcolor{Col2} {$ A = \left\{ \frac{1}{n} - 
          \frac{1}{m} \; : \; n,m \in \mathbb{N} \right\}$ }}
          Έχουμε ότι $ \sup A = 1 $  και $\inf A = -1 $.
          \begin{myitemize}
            \item Θα δείξουμε ότι $ \inf A = -1 $. Πράγματι:
              $ -1 $ κ.φ. του $A$, γιατί $ \frac{1}{n} - \frac{1}{m} \geq \frac{1}{n} 
              - 1 > -1, \; \forall n \in \mathbb{N} $.
              'Εστω $ \varepsilon > 0 $. Τότε, $ \exists n_{0} \in \mathbb{N} $ με 
              $ \frac{1}{n_{0}} < \varepsilon $ ώστε
              $ \frac{1}{n_{0}} - 1 < \varepsilon -1 $ με $ \frac{1}{n_{0}} -1 \in A $.
            \item Θα δείξουμε ότι $ \sup A = 1 $. Πράγματι:
              Παρατηρούμε ότι $ A = -A $, άρα από γνωστή πρόταση έχουμε ότι 
              \[ \inf A = - \sup (-A) = - \sup A \]
              Άρα \[ \sup A = - \inf A = -(-1) = 1 \]
          \end{myitemize}
      \end{enumerate}

    \item \textcolor{Col1}{Έστω $A$ φραγμένο υποσύνολο του $ \mathbb{R} $ τέτοιω ώστε 
      $ \sup A = \inf A $.  Να δείξετε ότι το $A$ είναι μονοσύνολο.}
      \begin{proof}
      \item {}
        Έστω $ c = \inf A = \sup A $. Τότε $ \forall a \in A $ ισχύει $ c \leq a \leq c
        $. Άρα $ A = \{ c \} $.
      \end{proof}

    \item \textcolor{Col1}{Αν $ A, B $ μη-κενά, φραγμένα υποσύνολα του 
        $ \mathbb{R} $ με $ A \subseteq B $ να δείξετε ότι 
      $ \inf B \leq \inf A \leq \sup A \leq \sup B $.}
      \begin{proof}
      \item {} 
        \begin{myitemize}
          \item Θα δείξουμε ότι $ \inf B \leq \inf A $. 

            Αρκεί να δείξουμε ότι $ \inf B $ κ.φ. του $A$. Πράγματι:

            Έστω $ x \in A \overset{A \subseteq B}{\Rightarrow} x \in B \Rightarrow 
            \inf B \leq x, \; \forall x \in A $. Άρα $ \inf B $ κ.φ. του $A$.
          \item Προφανώς $ \inf A \leq \sup A $, γιατί $ \forall x \in A, \; 
            \inf A \leq x \leq \sup A $.
          \item Θα δείξουμε οτι $ \sup A \leq \sup B $. 

            Άρκεί να δείξουμε ότι $ \sup B $ α.φ. του $A$. Πράγματι:

            Έστω $ x \in A \overset{A \subseteq B}{\Rightarrow} x \in B \Rightarrow 
            x \leq \sup B, \; \forall x \in A $.  Άρα $ \sup B $ α.φ. του $A$.
        \end{myitemize}
      \end{proof}

    \item \textcolor{Col1}{Έστω $ A, B $ μη-κενά υποσύνολα του $ \mathbb{R} $ 
        τέτοια ώστε $ a \leq b, \; \forall a \in A $ και $ \forall b \in B $.
        Να δείξετε ότι:
        \begin{enumerate}
          \item $ \sup A \leq b, \;  \forall b \in B $
          \item $ \sup A \leq \inf B $
      \end{enumerate}}
      \begin{proof}
      \item {}
        \begin{enumerate}
          \item \label{prop:one} Έστω $ b \in B \Rightarrow a \leq b, \; 
            \forall a \in A $. 
            Άρα $ b \in B $ α.φ. του $A, \; \forall b \in B$. Άρα 
            $ \sup A \leq b, \; \forall b \in B $.
          \item Από το \ref{prop:one} έχουμε ότι $ \sup A \leq b, \; \forall 
            b \in B$. Άρα το $ \sup A $ κ.φ. του $B$. Άρα $ \sup A \leq 
            \inf B$.
        \end{enumerate}
      \end{proof}

    \item \textcolor{Col1}{ Έστω $ A \subseteq \mathbb{R} $ μη-κενό, και κάτω 
        φραγμένο και έστω $ B $ το σύνολο των κάτω φραγμάτων του $A$. Να δείξετε ότι:
        \begin{enumerate}
          \item $ B \neq \emptyset $
          \item $B$ άνω φραγμένο.
          \item $ \sup B = \inf A $
      \end{enumerate}}
      \begin{proof}
      \item {}
        \begin{enumerate}
          \item $A$ κάτω φραγμένο. Άρα $ \exists x \in \mathbb{R} $ τ.ω. $x$ 
            κ.φ. του $A$, οπότε $ x \in B \Rightarrow B \neq \emptyset $.
          \item $ A \neq \emptyset $, άρα $ \exists x \in A $. Τότε $ \forall 
            y \in \mathbb{R}$ με $ y>x $ έχουμε ότι $ y $ όχι κ.φ. του A. 
            Άρα $ y \not\in b $. Άρα το $ B $ είναι άνω φραγμένο.
          \item 
            $  
            \left.
              \begin{tabular}{l}
                $ B \neq \emptyset $ \\
                $B$ άνω φραγμένο
              \end{tabular}
            \right\}
            \overset{\text{Α.Π.}}{\Rightarrow} B $ έχει supremum, έστω $ a = \sup B $.
            Τότε $ a \geq x, \; \forall x \in B $, άρα $ a \geq x, \; \forall x $
            όπου $x$ είναι κ.φ. του $A$, άρα $ a \geq \inf A $. Οπότε αρκεί να 
            δείξουμε ότι $ \inf A \geq a $, δηλ. αρκεί να δείξουμε ότι $ 
            a$ κ.φ. του $A$ (Με άτοπο). Πράγματι:

            Έστω ότι $ a $ όχι κ.φ. του $A$. Άρα $ \exists x \in A, \; x < a $. 
            Όμως $ a = \sup B $, οπότε από τη χαρ. ιδιοτ. του supremum έχουμε ότι
            $ \exists y \in B, \; x < y < a $. Άτοπο, γιατί $ y $ κ.φ. του $A$.

            Άρα $ \sup B = \inf A $.
        \end{enumerate}
      \end{proof}

    \item \textcolor{Col1}{\label{ask:3z} Να αποδείξετε ότι το σύνολο 
      $ A = \{ 3k \; : \; k \in \mathbb{Z} \} $ δεν είναι άνω φραγμένο.}
      \begin{proof}
      \item {}
        Πράγματι, έστω $ A $ άνω φραγμένο, και επειδή $A \neq \emptyset $ 
        υπάρχει το supremum του $A$, έστω $ \sup A = s $. Τότε $ s-1 < s $.
        Άρα από τη χαρ. ιδιοτ. του supremum, $ \exists k \in \mathbb{Z}, \; 
        3k > s-1$. Άρα $ 3k+3 > s-1 + 3 \Leftrightarrow 3(k+1) > s+2 > s $. Όμως 
        $ 3(k+1) \in \mathbb{Z} $, άτοπο, γιατί $ s = \sup A $.
      \end{proof}

    \item \textcolor{Col1}{Έστω $ A,B $ μη-κενά, φραγμένα υποσύνολα του $ \mathbb{R} $.
        Να δείξετε ότι:
        \begin{enumerate}
          \item $ A \cup B  $ είναι φραγμένο
          \item $ \sup {(A\cup B)} = \max \{ \sup A, \sup B \} $
          \item $ \inf {(A\cup B)} = \min \{ \inf A, \inf B \} $
          \item Ισχύει κάτι ανάλογο για το $ \sup {(A\cap B)} $ και 
            $ \inf {(A\cap B)} $;
      \end{enumerate}}
      \begin{proof}
      \item {}
        \begin{enumerate}
          \item {}
            $A$ φραγμένο $ \Leftrightarrow \exists M \in \mathbb{R} 
            > 0, \; -M < a < M, \; \forall a \in A $ 

            $B$ φραγμένο $ \Leftrightarrow \exists N \in \mathbb{R} 
            > 0, \; -N < b < N, \; \forall b \in B $ 

            Θέτουμε $ K = \max \{ M,N \} $. 

            Θα αποδείξουμε (με άτοπο) ότι $ -K < c < K, \; 
            \forall c \in A \cup B $.Πράγματι:

            Έστω $ c \in A \cup B $ με $ \abs{c} \geq K = \max \{ 
            M,N\} $. Τότε 

            $ 
            \left.
              \begin{tabular}{l}
                Αν $c \in A$ τότε $M > \abs{c} \geq \max \{ M,N \}$, 
                άτοπο \\
                Αν $c \in B$ τότε $N > \abs{c} \geq \max \{ M,N \}$,
                άτοπο
              \end{tabular} 
            \right\}  \Rightarrow \\
            -K < c < K, \; \forall c \in A \cup B $.
          \item 
            $
            \left.
              \begin{tabular}{l}
                $ A \neq \emptyset $, και άνω φραγμένο $
                \overset{\text{Α.Π.}}{\Rightarrow} \exists \sup A  $ \\

                $ B \neq \emptyset $, και άνω φραγμένο $
                \overset{\text{Α.Π.}}{\Rightarrow} \exists \sup B  $ \\
              \end{tabular}
            \right\}  \Rightarrow $ \\ 
            χ.β.γ. έστω $ \max \{ \sup A, \sup B \} = \sup A$. 

            Θα δείξουμε ότι $ \sup (A \cup B) = \sup A $. Πράγματι:

            Έστω $ c \in A \cup B $. Τότε, αν $ c \in A \Rightarrow c \leq 
            \sup A$, ενώ αν $ c \in B \Rightarrow c \leq \sup B \leq \sup A $. 
            Άρα σε κάθε περίπτωση $ c \leq \sup A, \; \forall c \in A \cup 
            B$. Άρα $ \sup A $ α.φ. του $ A \cup B $.

            Ισχύει ότι \inlineequation[eq:first]{ \sup (A \cup B) 
            \leq \sup A }, γιατί $ \sup A $ α.φ. 
            του $ A \cup B $. Θα δείξουμε ότι \inlineequation[eq:two]{ 
            \sup A \leq \sup (A \cup B) }.
            Πράγματι: 

            Αρκεί να δείξουμε ότι $ \sup (A \cup B) $ α.φ. του $A$. Πράγματι:

            έστω $ a \in A \Rightarrow a \in A \cup B \Rightarrow a \leq \sup 
            (A \cup B), \; \forall a \in A$, άρα το $ \sup A $ α.φ. του $A$.

            Οπότε από τις $ \eqref{eq:first} $ και $ \eqref{eq:two} $, 
            προκύπτει το ζητούμενο.
        \end{enumerate}
      \end{proof}

    \item \textcolor{Col1}{Έστω $ A, B $ μη-κενά, άνω φραγμένα υποσύνολα του
      $ \mathbb{R} $.}

      Αν $ A+B = \{ a+b \; : \; a \in A, \; b\in B \} $ και $A \cdot B = 
      \{ a\cdot b \; : \; a \in A, \; b \in B\}$ . Να δείξετε ότι 
      \begin{enumerate}
        \item $ \sup {(A+B)} = \sup A + \sup B $.
        \item $ \sup {(A\cdot B)} = \sup A \cdot \sup B $
      \end{enumerate}
      \begin{proof}
      \item {}
        $
        \left.
          \begin{tabular}{l}
            $ a \leq \sup A, \; \forall a \in A$ \\
            $ b \leq \sup B, \; \forall b \in B $
          \end{tabular}
        \right\} \Rightarrow a+b \leq \sup A + \sup B, \; \forall a \in A 
        $ και $ b \in B $.

        Οπότε $ \sup A + \sup B $ α.φ. του $ A + B \Rightarrow \sup (A+B) \leq 
        \sup A + \sup B$.
        Θ.δ.ο. $ \sup A + \sup B \leq \sup (A+B) $. Πράγματι:

        Έστω $ a \in A $ και $ b \in B \Rightarrow a+b \leq \sup (A+B) 
        \Leftrightarrow a \leq \sup (A+B) - b, \; \forall a \in A $ 

        Άρα $ \sup (A+B) - b $ α.φ. του $A, \; \forall b \in B $, 

        άρα $ \sup A \leq \sup (A+B) - b, \; \forall b \in B \Leftrightarrow 
        b \leq \sup (A+B) - \sup A , \; \forall b \in B$. 

        Άρα $ \sup (A+B) - \sup A $ α.φ. του $B$, και άρα 

        $ \sup B \leq \sup (A+B) - \sup A \Leftrightarrow \sup A + \sup B \leq 
        \sup (A+B)$.
      \end{proof}

    \item \textcolor{Col1}{Να αποδείξετε με χρήση της Μαθηματικής Επαγωγής 
        τους παρακάτω τύπους.  
        \begin{enumerate}
          \item $ 1^{2} + 2^{2} + \cdots + n^{2} = \frac{n(n+1)(2n+1)}{6}, \; 
            \forall n \in \mathbb{N} $
          \item $ 1^{3} + 2^{3} + \cdots + n^{3} = (1+2+\cdots + n)^{2}, \; 
            \forall n \in \mathbb{N} $
      \end{enumerate}}
      \begin{proof}
      \item {}
        \begin{enumerate}
          \item Για $ n=1 $, έχω: $ 1^{2} = \frac{1\cdot 2 \cdot 3}{6} = 1 $, 
            ισχύει.

            Έστω ότι ισχύει για $n=k$, δηλ, 
            $ \inlineequation[eq:epag1]{1^{2} + \cdots + k^{2} = 
            \frac{k (k+1)(2k+1)}{6}} $

            θ.δ.ο ισχύει και για $ n=k+1 $. Πράγματι:
            \begin{align*}
              1^{2} + \cdots + k^{2} + (k+1)^{2} 
                        &\overset{\eqref{eq:epag1}}{=}\frac{k(k+1)(2k+1)}{6} 
                        + (k+1)^{2} \\
                        &= \frac{k(k+1)(2k+1)+6(k+1)^{2}}{6} \\
                        &= \frac{(k+1)[k(2k+1)+6(k+1)]}{6} \\
                        &= \frac{(k+1)(2k^{2}+7k+6)}{6} \\
                        &= \frac{(k+1)2(k+2)(k+ \frac{3}{2})}{6} \\
                        &= \frac{(k+1)(k+2)(2k+3)}{6} \\
                        &= \frac{(k+1)[((k+1)+1)[2(k+1)+1]]}{6} 
            \end{align*}
          \item Για $ n=1 $, έχω: $ 1^{3} = 1^{2} $, ισχύει.

            Έστω ότι ισχύει για $ n=k $, δηλ. \inlineequation[eq:epag2]{1^{3} 
            + \cdots + k^{3} = (1+\cdots + k)^{2}}

            Θ.δ.ο. ισχύει και για $ n=k+1 $. Πράγματι:
            \begin{align*}
              [1+ \cdots &+ k + (k+1)]^{2} 
              = (1+\cdots + k)^{2} \\
                         & \hspace{0.5cm} + 2(1+\cdots + k)(k+1) + (k+1)^{2} = \\
                         &\overset{\eqref{eq:epag2}}{=} 1^{3}+\cdots +k^{3} + 2\cdot 
                         \frac{k(k+1)}{2}\cdot (k+1) + (k+1)^{2} \\
                         &=1^{3}+\cdots +k^{3}+(k+1)^{2}(k+1) \\
                         &= 1^{3}+\cdots +k^{3}+(k+1)^{3}
            \end{align*} 
        \end{enumerate}
      \end{proof}

    \item \label{ask:sums} \textcolor{Col1}{Βρείτε ένα κλειστό τύπο για τα 
        παρακάτω αθροίσματα: 
        \begin{enumerate}
          \item $ \sum_{k=1}^{n} (2k-1) = 1 + 3 + 5 + \cdots + (2n-1)  $
          \item $ \sum_{k=1}^{n} (2k-1)^{2} = 1^{2} + 3^{2} + 5^{2} 
            + \cdots + (2n-1)^{2}  $ 
      \end{enumerate}}
      \begin{proof}
      \item {}
        \begin{enumerate}
          \item \begin{align*}
              \sum_{k=1}^{n} (2k-1) &= 1 + 3 + \cdots + (2n-1) \\
                                    &=1 + 2 + 3 +\cdots +2n - 2(1+\cdots +n) \\
                                    &= \frac{2n(2n+1)}{2}-2 \cdot \frac{n(n+1)}{2} \\
                                    &=2n^{2}+n-n^{2}-n \\
                                    &=n^{2}
            \end{align*}
          \item \begin{align*}
              \sum_{k=1}^{n}
              & (2k-1)^{2} =1^{2}+3^{2}\cdots + (2n-1)^{2} = \\
              &=1^{2}+2^{2}+3^{2}+\cdots +(2n)^{2}-[2^{2}+4^{2}+6^{2}+\cdots +
              (2n)^{2}] \\
              &= \frac{2n(2n+1)(4n+1)}{6} - 4[1^{2}+2^{2}+3^{2}+\cdots +n^{2}] \\
              &= \frac{2n(2n+1)(4n+1)}{6} - 4 \cdot \frac{n(n+1)(2n+1)}{6} \\
              &= \frac{2n(2n+1)[(4n+1)-2(n+1)]}{6} \\
              &= \frac{n(2n+1)(2n-1)}{3} 
            \end{align*}
        \end{enumerate}
      \end{proof}

    \item \label{ask:thema18sum} \textcolor{Col1}{({\bfseries Θέμα: 2018}) 
        Να αποδείξετε ότι $ \sum_{n=2}^{N-2} \frac{1}{(n+1)(n+2)} 
      = \frac{1}{3} - \frac{1}{N} $}
      \begin{proof}
      \item {}
        $ a_n = \frac{1}{(n+1)(n+2)} = \frac{A}{n+1} + \frac{B}{n+2} = 
        \frac{A(n+2)+B(n+1)}{(n+1)(n+2)} $

        Άρα $A(n+2) = B(n+1) = 1, \; \forall n \in \mathbb{N}$

        Δηλ. $ (A+B)n+2A+B=1, \; \forall n \in \mathbb{N} $, οπότε πρέπει:

      $  \sysdelim.\}\systeme{A+B=0,2A+B=1} \Leftrightarrow \sysdelim..
      \systeme{A=1,B=-1} $

      Οπότε $ a_{n} = \frac{1}{(n+1)(n+2)} = \frac{1}{n+1} - \frac{1}{n+2} $

      Άρα 
      \begin{align*}
        &\sum_{n=2}^{N-2} \frac{1}{(n+1)(n+2)} 
        = \sum_{n=2}^{N-2} \left(\frac{1}{n+1} - \frac{1}{n+2}\right) = \\
        &= \frac{1}{2+1} - \frac{1}{2+2} + \frac{1}{3+1} - \frac{1}{3+2} +\cdots +
        \frac{1}{N-2+1} - \frac{1}{N-2+2} \\
        &= \frac{1}{3} - \frac{1}{N} 
      \end{align*}
    \end{proof}

  \item \textcolor{Col1}{Να αποδείξετε ότι $ n^{5} - n $ είναι 
    πολλαπλάσιο του 5, $ \forall n \in \mathbb{N} $.}
    \begin{proof}
      Για $ n=1 $, έχω: $ 1-1=0 $, είναι πολ/σιο του 5, γιατί $0=0\cdot 5$.

      Έστω ότι ισχύει για $n$, δηλ. $n^{2}-n $ 
      είναι πολ/σιο του 5 $
      \Leftrightarrow \inlineequation[eq:epag3]{n^{2}-n=5k, 
      \; k \in \mathbb{Z}} $.

      Θ.δ.ο. ισχύει και για (n+1). Πράγματι:
      \begin{align*}
        (n+1)^{5}-(n+1) &= n^{5}+5n^{4}+10n^{3}+10n^{2}+5n+1-n+1 \\
                        &=n^{5}-n+5(n^{4}+2n^{3}+2n^{2}+n) \\
                        &\overset{\eqref{eq:epag3}}{=} 5k 
                        + 5(n^{4}+2n^{3}+2n^{2}+n) \\ 
                        &= 5(k + n^{4}+2n^{3}+2n^{2}+n)
      \end{align*}
    \end{proof}

  \item \textcolor{Col1}{Να αποδείξετε ότι $ n! > 2^{n}, \forall n \geq 4 $}
    \begin{proof}
    \item {}
      Για $ n=4 $, έχω: $ 4! = 1\cdot 2 \cdot 3 \cdot 4 = 24 > 16 = 2^{4} $

      Έστω ότι ισχύει για $n$, δηλ $ n! >2^{n} $. 

      Θ.δ.ο. ισχύει και για $ n+1 $. Πράγματι:

      \begin{align*} (n+1)! = 1 \cdot 2 \cdots n \cdot (n+1)  
        =n!(n+1) > 2^{n}(n+1) \geq 2^{n}\cdot 2 = 2^{(n+1)}
      \end{align*}
    \end{proof}

  \item \textcolor{Col1}{Έστω $ a \in \mathbb{R} $ και $ n \in \mathbb{N}$.
      Να δείξετε με τη βοήθεια της μαθηματικής επαγωγής της 
      παρακάτω ανισότητες.
      \begin{enumerate}
        \item Αν $ 0<a< \frac{1}{n} $ τότε $ (1+a)^{n} < \frac{1}{1-na} $
        \item Αν $ 0 \leq a \leq 1$  τότε $ 1-na \leq (1-a)^{n} \leq
          \frac{1}{1+na} $
    \end{enumerate}}
    \begin{proof}
    \item {}
      \begin{enumerate}
        \item Για $ n=1 $, έχω: $ 1+a < \frac{1}{1-a} 
          \Leftrightarrow (1+a)(1-a) < 1 
          \Leftrightarrow 1-a^{2} <1 \Leftrightarrow a^{2} > 0  $,
          ισχύει για $ 0<a<1 $.

          Έστω ισχύει για $ n $, δηλ. $ (1+a)^{n} < \frac{1}{1-na}
          $, για $ 0 < a < \frac{1}{n} $.

          Θ.δ.ο. ισχύει για $ n+1 $. Πράγματι:
          \begin{align*}
            (1+a)^{n+1} &= (1+a)(1+a)^{n} \\
                        &< (1+a)\cdot \frac{1}{1-na} \\
                        &= \frac{(1+a)(1-a)}{(1-na)(1-a)} \\
                        &= \frac{1-a^{2}}{(1-na)(1-a)} \\ 
                        &= \frac{1-a^{2}}{1-na-a+na^{2}} \\
                        &= \frac{1-a^{2}}{1-a(n+1)+na^{2}} \\
                        &< \frac{1-a^{2}}{1-a(n+1)} \\
                        &< \frac{1}{1-a(n+1)} 
          \end{align*}
        \item Αποδεικνύουμε πρώτα ότι $ (1-a)^{n} \leq \frac{1}{1+na}$,
          αν $ 0 \leq a \leq 1 $.

          Για $ n=1 $, έχω: $ 1-a \leq \frac{1}{1+a} 
          \Leftrightarrow (1-a)(1+a) \leq 1 \Leftrightarrow 
          1-a^{2} \leq 1 \Leftrightarrow a^{2} \geq 0  $, ισχύει
          για $ 0 \leq a \leq 1 $. 

          Έστω ότι ισχύει για $n$, δηλ. $ (1-a)^{n} \leq
          \frac{1}{1+na} $, αν $ 0 \leq a \leq 1 $.

          θ.δ.ο. ισχύει για $ n+1 $. Πράγματι: 
          \begin{align*}
            (1-a)^{n+1} &= (1-a)^{n}(1-a) \\
                        &\leq \frac{1}{1+na} \cdot (1-a) \\
                        &= \frac{(1-a)(1+a)}{(1+na)(1+a)} \\
                        &= \frac{1-a^{2}}{1+(n+1)a +na^{2}} \\
                        &< \frac{1-a^{2}}{1 + (n+1)a} \\
                        &< \frac{1}{1 + (n+1)a} 
          \end{align*}
          Αποδεικνύουμε, τώρα, ότι $ 1-na \leq (1-a)^{n} $, αν 
          $ 0 \leq a \leq 1 $. 

          Για $ n=1 $, έχω: $ 1-a \leq 1-a $, ισχύει.

          Έστω ότι ισχύει για $n$, δηλ. $ 1-na \leq (1-a){n} $.

          θ.δ.ο. ισχύει και για $ n+1 $. Πράγματι:
          \begin{align*}
            (1-a)^{n+1}&=(1-a)^{n}\cdot (1-a) \\
                       &\geq (1-na)(1-a) \\
                       &=1-a-na+na^{2} \\
                       &= 1-(n+1)a + na^{2} \\
                       &\geq 1-(n+1)a
          \end{align*}
      \end{enumerate}
    \end{proof}

  \item \textcolor{Col1}{Αν $a > 0$ τότε να αποδείξετε ότι $ (1+a)^{n} 
    \geq 1 + na + \frac{n(n-1)a^{2}}{2},\; \forall n \in \mathbb{N}$}
    \begin{proof}
    \item {}
      Για $ n=1 $, έχω: $ 1+a \geq 1+a $, ισχύει.

      Έστω ότι ισχύει για $n$, δηλ. $ (1+a)^{n} \geq 1+na +
      \frac{n(n-1)a^{2}}{2} $.

      θ.δ.ο. ισχύει και για $ n+1 $. Πράγματι:
      \begin{align*}
        (1+a)^{n+1} &= (1+a)^{n}\cdot (1+a) \geq (1+na+ 
        \frac{n(n-1)a^{2}}{2})(1+a) \\
                    &= 1+na+ \frac{n(n-1)a^{2}}{2} + a + na^{2} + 
                    \frac{n(n-1)a^{3}}{2} \\
                    &= 1+(n+1)a+ \frac{n^{2}-n+2n}{2} a^{2} + 
                    \frac{n(n-1)}{2} a^{3} \\
                    &\overset{a>0}{>} 1+(n+1)a + \frac{n(n+1)}{2} a^{2}
      \end{align*}
    \end{proof}

  \item \textcolor{Col1}{Να δείξετε ότι οι παρακάτω αριθμοί είναι άρρητοι.
      \begin{enumerate}
        \item $ \sqrt{3} $ και  $ \sqrt{5} $
        \item $ \sqrt[3]{2} $ και $ \sqrt[3]{3} $
        \item $ \sqrt{2} + \sqrt{3} $ και $ \sqrt{2} + \sqrt{6} $ 
    \end{enumerate}}
\end{enumerate}
\begin{proof}
\item {}
  \begin{enumerate}[i)]
    \item Έστω $ \sqrt{2} + \sqrt{6} $ ρητός. Άρα $ (\sqrt{2} +
      \sqrt{6} )^{2} = 2 + 2 \sqrt{12} + 6 = 8 + 4 \sqrt{3} $ 
      ρητός, που σημαίνει ότι $ \sqrt{3} $ είναι ρητός (
      θυμάμαι ότι ρητός + άρρητος = άρρητος). Άτοπο, 
      γιατί $ \sqrt{3} $ άρρητος.
  \end{enumerate}
\end{proof}
\end{multicols}


\end{document}

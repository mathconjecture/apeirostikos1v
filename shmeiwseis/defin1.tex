\input{preamble.tex}
\input{definitions.tex}



\begin{document}

\pagestyle{vangelis}

\begin{center}
  \minibox[c]{\large \bfseries \textcolor{Col1}{Πραγματικοί Αριθμοί}}
\end{center}

\vspace{\baselineskip}

\begin{dfn}
    $ A \subseteq \mathbb{R}, \; A \neq \emptyset $ \textcolor{Col1}{άνω 
    φραγμένο} $ \Leftrightarrow \exists M \in \mathbb{R} \; 
    \text{(\textcolor{Col1}{άνω φράγμα})} \; : a \leq M, \; 
    \forall a \in A$ 
\end{dfn}

\begin{dfn}
    $ A \subseteq \mathbb{R}, \; A \neq \emptyset $ \textcolor{Col1}{κάτω 
    φραγμένο} $ \Leftrightarrow \exists m \in \mathbb{R} \; 
    \text{(\textcolor{Col1}{κάτω φράγμα})}  \; : a \geq m, \; 
    \forall a \in A$
\end{dfn}

\begin{dfn}
    $ A \subseteq \mathbb{R}, \; A \neq \emptyset $ \textcolor{Col1}{ 
    φραγμένο}  \begin{tabular}[t]{l} $ \Leftrightarrow A $ είναι άνω και 
        κάτω φραγμένο \\
        $ \Leftrightarrow \exists m,M \in \mathbb{R} \; : \; m \leq a \leq M,
        \; \forall a \in A $
    \end{tabular}
\end{dfn}

\begin{dfn}
        $ A \subseteq \mathbb{R}, \; A \neq \emptyset $ \textcolor{Col1}{ 
        απολύτως φραγμένο} \begin{tabular}[t]{l}
        $\Leftrightarrow \exists M>0 \; : \; \abs{a} \leq M, \; \forall a 
        \in A $ \\
        $ \Leftrightarrow \exists M>0  \; : \; 
        -M \leq a \leq M, \; \forall a \in A $
        \end{tabular}
\end{dfn}

\begin{rem}
\item {}
    \begin{itemize}[label=\textcolor{Col1}{\tiny$\blacksquare$}]
        \item Αν $ M $ άνω φράγμα του $A$ και $ \textcolor{Col2}{M \in A} $ τότε $M$ λέγεται 
    \textcolor{Col1}{μέγιστο} του $A$ και συμβολίζεται $ M = \max A $.
\item Αν $ m $ κάτω φράγμα του $A$ και $ \textcolor{Col2}{m \in A} $ τότε $m$ λέγεται \textcolor{Col1}{ελάχιστο} του $A$ και συμβολίζεται $ m = \min A $.  \end{itemize}
\end{rem}

\begin{prop}
    $ A \subseteq \mathbb{R}, \; A \neq \emptyset $ είναι φραγμένο 
    $ \Leftrightarrow $ $ A $ είναι απολύτως φραγμένο.
\end{prop}


\begin{dfn}
    Έστω $ A \subseteq \mathbb{R}, \; A \neq \emptyset $.
    \begin{itemize}[label=\textcolor{Col1}{\tiny$\blacksquare$}]
        \item $ M \in \mathbb{R} $ \textcolor{Col1}{supremum} του $A$ (
            $ \textcolor{Col2}{M = \sup A} $ ) $
            \Leftrightarrow \begin{tabular}[t]{l}
                M άνω φράγμα του A $\Leftrightarrow a \leq M, \; \forall a 
                \in A $ \\
                M ελάχιστο άνω φράγμα του A $ \Leftrightarrow M' 
                \text{α.φ. του} A \Rightarrow M \leq M' $
            \end{tabular} $

        \item $ m \in \mathbb{R} $ \textcolor{Col1}{infimum} του $A$ 
            ($ \textcolor{Col2}{m= \inf A} $) $
            \Leftrightarrow \begin{tabular}[t]{l}
                m κάτω φράγμα του A $ \Leftrightarrow a \geq m, \; \forall a 
                \in A $ \\
                M μέγιστο κάτω φράγμα του A $ \Leftrightarrow m' 
                \text{κ.φ. του} A \Rightarrow m \geq m' $
            \end{tabular} $
    \end{itemize}
\end{dfn}

\begin{rem}
\item {}
    \begin{itemize}[label=\textcolor{Col1}{\tiny$\blacksquare$}]
        \item 
    Η χαρακτηριστική ιδιότητα που έχει το supremum ενός συνόλου $ A $, είναι
    ότι είναι το ελάχιστο από τα άνω φράγματα του $A$. Δηλαδή, με άλλα λόγια,
    κάθε πραγματικός αριθμός μικρότερος του $M$ δεν είναι άνω φράγμα του 
    $A$, επομένως, $ \forall k < M, \; \exists a \in A \; : \; k < a < M $, 
    ή καλύτερα, $ \forall \varepsilon > 0, \; \exists a \in A \; : \; 
    M- \varepsilon < a$

\item 
    Η χαρακτηριστική ιδιότητα που έχει το infimum ενός συνόλου $ A $, είναι
    ότι είναι το μέγιστο από τα κάτω φράγματα του $A$. Δηλαδή, με άλλα λόγια,
    κάθε πραγματικός αριθμός μεγαλύτερος του $m$ δεν είναι κάτω φράγμα του 
    $A$, επομένως, $ \forall k > m, \; \exists a \in A \; : \; m < a < k $, 
    ή καλύτερα, $ \forall \varepsilon > 0, \; \exists a \in A \; : \; 
    a < m + \varepsilon $
    \end{itemize}
\end{rem}
\begin{thm}[\textcolor{Col2}{Χαρακτηριστική ιδιότητα του sup και inf}]
\item {}
    \begin{itemize}[label=\textcolor{Col1}{\tiny$\blacksquare$}]
        \item Έστω $ A \subseteq \mathbb{R}, \; A \neq \emptyset $, $ A $ άνω φραγμένο με $M \in \mathbb{R}$ α.φ. του A, τότε 
    \[
         M = \sup A \Leftrightarrow \forall \varepsilon > 0, \; \exists 
         a \in A \; : \; M- \varepsilon  < a
     \] 

 \item Έστω $ A \subseteq \mathbb{R}, \; A \neq \emptyset $, $ A $ κάτω
    φραγμένο με $m \in \mathbb{R}$ κ.φ. του A, τότε 
    \[
         m = \inf A \Leftrightarrow \forall \varepsilon > 0, \; \exists 
         a \in A \; : \; a < m+ \varepsilon 
     \] 
    \end{itemize}
    \end{thm}

\begin{prop}
        \item {}
    \begin{itemize}[label=\textcolor{Col1}{\tiny$\blacksquare$}]
        \item $ A \subseteq \mathbb{R}, \; A \neq \emptyset $ και $A$ όχι 
            άνω φραγμένο $ \Rightarrow \sup A = + \infty $.
        \item $ A \subseteq \mathbb{R}, \; A \neq \emptyset $ και $A$ όχι 
            κάτω φραγμένο $ \Rightarrow \inf A = - \infty $.
    \end{itemize}

    

\end{prop}
\end{document}
